\documentclass[12pt]{article}
\documentclass[12pt]{article}

\usepackage[paper=letterpaper, margin=0.5in, footskip=0.25in]{geometry}
\usepackage{amsmath}
\usepackage{amssymb}
\usepackage{titlesec}
\usepackage{enumitem}
\usepackage{graphicx}
\usepackage{framed}
\usepackage{xcolor}
\usepackage{pgfplots}
\usetikzlibrary{calc,karnaugh,positioning}
\usepackage[american]{circuitikz}
\usepackage{siunitx}
\usepackage{changepage}
\usepackage{datetime}
\usepackage[T1]{fontenc}
\usepackage{makecell}
\usepackage{multicol}
\usepackage{steinmetz}

\pgfplotsset{compat=newest}

\tikzset{every karnaugh/.style={American style,kmlabel left/.style={black, font=\small, left=2pt}, kmlabel top/.style={black, font=\small, above=1pt}}}
\tikzset{
	grp/.style n args={3}{#1,
		thick,
		minimum width=#2\kmunitlength,
		minimum height=#3\kmunitlength,
		rounded corners=0.2\kmunitlength,
		rectangle,draw}
}

\newdateformat{monthyeardate}{\monthname[\THEMONTH] \THEDAY, \THEYEAR}

\titleformat{\section}{}{\large\textbf\thesection}{0em{\large\textbf. }}{\large\textbf}
\titleformat{\subsection}{}{\thesubsection}{0em{: }}{\textsc}
\titlespacing*{\section}{0em}{1em}{0.25em}
\titlespacing*{\subsection}{0em}{1em}{0.25em}

\makeatletter
\def\@maketitle{
	\centering
	\LARGE \@title
	\par
	\Large \@author
	\par
	\monthyeardate\today
	\vskip 1em
}\makeatother

\DeclareMathOperator{\union}{\cup}
\DeclareMathOperator{\intersect}{\cap}
\DeclareMathOperator{\var}{Var}

\newcommand{\V}[1]{\text{V\textsubscript{#1}}}
\newcommand{\Vv}[1]{\text{\textit{v}\textsubscript{#1}}}

\newcommand{\I}[1]{\text{I\textsubscript{#1}}}
\newcommand{\Ii}[1]{\text{\textit{i}\textsubscript{#1}}}

\newcommand{\Pp}[1]{\text{P\textsubscript{#1}}}

\newcommand{\U}[1]{\text{U\textsubscript{#1}}}
\newcommand{\E}[1]{\text{E\textsubscript{#1}}}

\newcommand{\A}[1]{\text{A\textsubscript{#1}}}

\newcommand{\Z}[1]{\text{Z\textsubscript{#1}}}
\newcommand{\R}[1]{\text{R\textsubscript{#1}}}
\newcommand{\C}[1]{\text{C\textsubscript{#1}}}
\newcommand{\Ll}[1]{\text{L\textsubscript{#1}}}

\newcommand{\qv}[1]{\qty{#1}{\volt}}
\newcommand{\qkv}[1]{\qty{#1}{\kilo\volt}}
\newcommand{\qmv}[1]{\qty{#1}{\milli\volt}}
\newcommand{\quv}[1]{\qty{#1}{\micro\volt}}
\newcommand{\qnv}[1]{\qty{#1}{\nano\volt}}
\newcommand{\qMv}[1]{\qty{#1}{\mega\volt}}

\newcommand{\qa}[1]{\qty{#1}{\ampere}}
\newcommand{\qma}[1]{\qty{#1}{\milli\ampere}}
\newcommand{\qua}[1]{\qty{#1}{\micro\ampere}}
\newcommand{\qna}[1]{\qty{#1}{\nano\ampere}}

\newcommand{\qw}[1]{\qty{#1}{\watt}}
\newcommand{\qkw}[1]{\qty{#1}{\kilo\watt}}
\newcommand{\qmw}[1]{\qty{#1}{\milli\watt}}
\newcommand{\quw}[1]{\qty{#1}{\micro\watt}}
\newcommand{\qnw}[1]{\qty{#1}{\nano\watt}}
\newcommand{\qMw}[1]{\qty{#1}{\mega\watt}}

\newcommand{\qj}[1]{\qty{#1}{\joule}}
\newcommand{\qkj}[1]{\qty{#1}{\kilo\joule}}
\newcommand{\qmj}[1]{\qty{#1}{\milli\joule}}
\newcommand{\quj}[1]{\qty{#1}{\micro\joule}}
\newcommand{\qnj}[1]{\qty{#1}{\nano\joule}}
\newcommand{\qMj}[1]{\qty{#1}{\mega\joule}}

\newcommand{\qr}[1]{\qty{#1}{\ohm}}
\newcommand{\qkr}[1]{\qty{#1}{\kilo\ohm}}
\newcommand{\qmr}[1]{\qty{#1}{\milli\ohm}}
\newcommand{\qur}[1]{\qty{#1}{\micro\ohm}}
\newcommand{\qnr}[1]{\qty{#1}{\nano\ohm}}
\newcommand{\qMr}[1]{\qty{#1}{\mega\ohm}}

\newcommand{\qf}[1]{\qty{#1}{\farad}}
\newcommand{\qmf}[1]{\qty{#1}{\milli\farad}}
\newcommand{\quf}[1]{\qty{#1}{\micro\farad}}
\newcommand{\qnf}[1]{\qty{#1}{\nano\farad}}
\newcommand{\qpf}[1]{\qty{#1}{\pico\farad}}

\newcommand{\qh}[1]{\qty{#1}{\henry}}
\newcommand{\qmh}[1]{\qty{#1}{\milli\henry}}
\newcommand{\quh}[1]{\qty{#1}{\micro\henry}}
\newcommand{\qnh}[1]{\qty{#1}{\nano\henry}}

\class{ECE 147B}
\title{Lab $n$ Report Template}
\author{Erk Sampat, Robert Kastaman, \ldots}
\begin{document}
\maketitle
\section{Abstract}
\label{sec:abstract}
The abstract should clearly summarize \textit{all} sections. Refer to sections like this: \autoref{sec:prelab}. Refer to subsections like this: \autoref{subsec:procedure_wk1}.
\section{Introduction}
\label{sec:intro}
The introduction is simply a more detailed version of the abstract.
\section{Pre-Lab Exercises}
\label{sec:prelab}
\subsection{Week 1}
\label{subsec:prelab_wk1}
Probably some boring math here\ldots
\subsection{Week 2}
\label{subsec:prelab_wk2}
and here as well. Align equations with equal signs like this:
\begin{align*}
	1 &= 1 \\
	2 &= 2
\end{align*}
If desired, enumerate steps like this:
\begin{enumerate}
	\item Justify your answer; answers with no explanation will receive no credit.
	\begin{enumerate}
		\item Is the system $y(t)=\int_{0}^{t+1}x(\tau-1)d\tau$ causal?
		\item Is the system $y(t)=tx(t^{2})$ linear?
	\end{enumerate}
	\item Consider the differential equation $y''+y=x$.
	\begin{enumerate}
		\item Determine the corresponding transfer function.
		\item Determine the state-space representation of the system.
	\end{enumerate}
\end{enumerate}
This style of enumeration may be used throughout the report.
\section{Procedure}
\label{sec:procedure}
\subsection{Week 1}
\label{subsec:procedure_wk1}
Code-style text can be written like this: \texttt{rlocfind}. Refer to figures like this: \autoref{fig:example}. All figures must be in the \texttt{figs} directory.
\begin{center}
    \includegraphics[width=0.25\linewidth]{example-image-a}
    \captionof{figure}{Example figure with a single graphic. It should usually be bigger (around \texttt{0.6\textbackslash linewidth}).}
    \label{fig:example}
\end{center}
Refer to code fragments like this: \autoref{cf:example}.
\begin{center}
    \begin{lstlisting}
        s = tf('s');
        G = (s^4+3*s^2+2)/(6*s+9);
        figure;
        margin(G);
    \end{lstlisting}
    % alternatively, use \lstinputlisting{}. the input path is ``code''. you can use this to directly include a matlab file.
    \captionof{codefrag}{Example code fragment.}
    \label{cf:example}
\end{center}
\subsection{Week 2}
\label{subsec:procedure_wk2}
Rest of the procedure goes here.
\section{Results}
\label{sec:results}
\subsection{Week 1}
\label{subsec:results_wk1}
Below is a crucial example of a figure with multiple graphics, each in its own subfigure:
\begin{figure}[H]
	\centering
	\subfloat[Subfigure (a)]{
		\label{subfig:example_a}
		\centering
		\includegraphics[width=0.45\linewidth]{example-image-a}
	}\hfill
	\subfloat[Subfigure (b)]{
		\label{subfig:example_b}
		\centering
		\includegraphics[width=0.45\linewidth]{example-image-b}
	}\\
	\caption{Example figure with two graphics, each in its own subfigure.}
	\label{fig:example2}
\end{figure}
\noindent Refer to the overall figure as before: \autoref{fig:example2}. Refer to each subfigure like this: \autoref{subfig:example_a}, \autoref{subfig:example_b}.
\subsection{Week 2}
\label{subsec:results_wk2}
Probably many, many more plots here\ldots
\section{Discussion}
\label{sec:discussion}
\subsection{Week 1}
\label{subsec:discussion_wk1}
This is the part where you turn off your brain and ask ChatGPT to explain your results.
\subsection{Week 2}
\label{subsec:discussion_wk2}
Here as well. And you're done!
\end{document}