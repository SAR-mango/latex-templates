\documentclass[12pt]{article}

\usepackage[paper=letterpaper,margin=0.5in,footskip=0.25in]{geometry}
\usepackage{amsmath}
\usepackage{amssymb}
\usepackage{titlesec} % for custom title format
%\usepackage{enumitem} % for lists
%\usepackage{graphicx} % for including images, pdfs, etc.
%\usepackage{framed}
%\usepackage[dvipsnames]{xcolor} % for additional colors
%\usepackage{pgfplots} % for plots
%\usetikzlibrary{calc,positioning,arrows.meta} % for tikz positioning help and arrows
%\usetikzlibrary{karnaugh} % for k-maps
%\usepackage[american]{circuitikz} % for circuits!
\usepackage[per-mode=fraction,inter-unit-product=\ensuremath{{}\cdot{}}]{siunitx} % for nicely formatted units
\usepackage{changepage} % for problem/part/subpart formatting
\usepackage{datetime} % for custom date format
\usepackage[T1]{fontenc} % everyone recommends to use this but idk what it's for
\usepackage{mathtools} % for the Aboxed command, which lets you box things within the align environment
%\usepackage{makecell} % for more complicated table formatting
%\usepackage{steinmetz} % for the phase ("angle of") symbol
%\usepackage{calc} % to perform arithmetic on spacing units

%\pgfplotsset{compat=newest}
%\tikzset{>=Stealth}
%\pgfplotsset{
%	standard/.style={
%		trig format=rad,
%		axis x line=middle,
%		axis y line=middle,
%		x label style={right},
%		y label style={above},
%		y tick label style={right},
%		axis line style={-Stealth,semithick},
%		major tick style={thick,black},
%		extra x tick labels={\empty},
%		extra y tick labels={\empty}
%	}
%}

% to format matrices like tables; for example, [cc|c]. this lets you create an augmented matrix.
%\makeatletter
%\renewcommand*\env@matrix[1][*\c@MaxMatrixCols c]{%
%	\hskip -\arraycolsep
%	\let\@ifnextchar\new@ifnextchar
%	\array{#1}}
%\makeatother

% k-map configuration
%\tikzset{every karnaugh/.style={American style,kmlabel left/.style={black, font=\small, left=2pt}, kmlabel top/.style={black, font=\small, above=1pt}}}
%\tikzset{
%	grp/.style n args={3}{#1,
%		thick,
%		minimum width=#2\kmunitlength,
%		minimum height=#3\kmunitlength,
%		rounded corners=0.2\kmunitlength,
%		rectangle,draw}
%}

% date format
\newdateformat{monthyeardate}{\monthname[\THEMONTH] \THEDAY, \THEYEAR}

% problem/part/subpart formatting
\newcommand{\prob}{\section{}}
\renewcommand{\thesubsection}{\alph{subsection}}
\renewcommand{\thesubsubsection}{\roman{subsubsection}}
\newenvironment{pp}{\adjustwidth{\parindent}{\parindent}\mbox{}\vspace{-\baselineskip}\subsection{}}{\endadjustwidth}
\newenvironment{spart}{\adjustwidth{\parindent}{\parindent}\mbox{}\vspace{-\baselineskip}\subsubsection{}}{\endadjustwidth}

% redefine sections/subsections/subsubsections for problem/part/subpart
\titleformat{\section}{}{\large\textbf{Problem \thesection.}}{0em}{}
\titleformat{\subsection}[runin]{}{\textbf{(\thesubsection)}}{0em}{}
\titleformat{\subsubsection}[runin]{}{\thesubsubsection.}{0em}{}
\titlespacing*{\section}{0em}{1em}{0em}
\titlespacing*{\subsection}{0em}{0.5em}{1ex}
\titlespacing*{\subsubsection}{0em}{0.5em}{1ex}

% title format
\makeatletter
\def\@maketitle{
	\centering
	\LARGE \@title
	\par
	\Large \@author
	\par
	\monthyeardate\today
	\vskip 1em
}\makeatother
